\documentclass[fleqn]{article}
\usepackage[left=1in, right=1in, top=1in, bottom=1in]{geometry}
\usepackage{mathexam}
\usepackage{verbatim}

\ExamClass{CSCE 240}
\ExamName{Homework 3}
\ExamHead{Due: 17 March 2020}

\let
\ds
\displaystyle

\begin{document}
\ExamInstrBox {
  You shall submit a zipped, \textbf{and only zipped}, archive of your homework
  directory, hw3. The directory shall contain, at a minimum, the files
  \texttt{src/sum\_finder.cc} and \texttt{inc/sum\_finder.h}. Name the archive
  submission file hw3.zip \\

  I will use my own makefile to compile and link to your
  \texttt{src/sum\_finder.cc} and \texttt{inc/sum\_finder.h} files. You must
  submit, at least, these two files.
}
%
Sometimes when I sit in restaurants waiting on food, I request the children's
menu to play the games. One of my favorites is the word puzzle in which you
search for words hidden in a scramble of letters. I began to work out a
solution to this game programmatically. That seemed a little simple, so I
decided to have you solve the problem of a scramble of integers, finding the
sub-row or sub-column which sums to another integer.\\
%
\\
You will provide me with a library containing (at least) the function
\texttt{FindSum} as demonstrated in the file \texttt{sum\_finder.cc}. You
should read the header file as well as my test file to understand what is
expected from the library.  Do note that the preconditions keep you from any
unpleasantness due to array bounds and the like. \\
%
\\
I have provided you a basic test application which you can use to ensure that
your code is, at least partially, correct. I would suggest a more rigorous
testing scheme to ensure that your methods handle any edge cases. \\
%
\\
Late assignments will lose 10\% per day late, with no assignment begin accepted
after 3 days (at 100\% reduction in points).\\
\\
%
You will receive 1.5 points for each if your \texttt{FindSum} function finds:
\begin{itemize}
  \item horizontal summation left-to-right,
  \item horizontal summation right-to-left,
  \item vertical summations top-to-bottom,
  \item vertical summations bottom-to-top,
  \item that the summation does not exist (2 points)
\end{itemize}
\end{document}

